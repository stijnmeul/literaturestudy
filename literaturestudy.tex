%% bare_jrnl.tex
%% V1.4
%% 2012/12/27
%% by Michael Shell
%% see http://www.michaelshell.org/
%% for current contact information.
%%
%% This is a skeleton file demonstrating the use of IEEEtran.cls
%% (requires IEEEtran.cls version 1.8 or later) with an IEEE journal paper.
%%
%% Support sites:
%% http://www.michaelshell.org/tex/ieeetran/
%% http://www.ctan.org/tex-archive/macros/latex/contrib/IEEEtran/
%% and
%% http://www.ieee.org/



% *** Authors should verify (and, if needed, correct) their LaTeX system  ***
% *** with the testflow diagnostic prior to trusting their LaTeX platform ***
% *** with production work. IEEE's font choices can trigger bugs that do  ***
% *** not appear when using other class files.                            ***
% The testflow support page is at:
% http://www.michaelshell.org/tex/testflow/


%%*************************************************************************

%% Legal Notice:
%% This code is offered as-is without any warranty either expressed or
%% implied; without even the implied warranty of MERCHANTABILITY or
%% FITNESS FOR A PARTICULAR PURPOSE! 
%% User assumes all risk.
%% In no event shall IEEE or any contributor to this code be liable for
%% any damages or losses, including, but not limited to, incidental,
%% consequential, or any other damages, resulting from the use or misuse
%% of any information contained here.
%%
%% All comments are the opinions of their respective authors and are not
%% necessarily endorsed by the IEEE.
%%
%% This work is distributed under the LaTeX Project Public License (LPPL)
%% ( http://www.latex-project.org/ ) version 1.3, and may be freely used,
%% distributed and modified. A copy of the LPPL, version 1.3, is included
%% in the base LaTeX documentation of all distributions of LaTeX released
%% 2003/12/01 or later.
%% Retain all contribution notices and credits.
%% ** Modified files should be clearly indicated as such, including  **
%% ** renaming them and changing author support contact information. **
%%
%% File list of work: IEEEtran.cls, IEEEtran_HOWTO.pdf, bare_adv.tex,
%%                    bare_conf.tex, bare_jrnl.tex, bare_jrnl_compsoc.tex,
%%                    bare_jrnl_transmag.tex
%%*************************************************************************

% Note that the a4paper option is mainly intended so that authors in
% countries using A4 can easily print to A4 and see how their papers will
% look in print - the typesetting of the document will not typically be
% affected with changes in paper size (but the bottom and side margins will).
% Use the testflow package mentioned above to verify correct handling of
% both paper sizes by the user's LaTeX system.
%
% Also note that the "draftcls" or "draftclsnofoot", not "draft", option
% should be used if it is desired that the figures are to be displayed in
% draft mode.
%
\documentclass[journal]{IEEEtran}
%
% If IEEEtran.cls has not been installed into the LaTeX system files,
% manually specify the path to it like:
% \documentclass[journal]{../sty/IEEEtran}

\bibliographystyle{IEEEtran}
\usepackage[none]{hyphenat}

% Some very useful LaTeX packages include:
% (uncomment the ones you want to load)

% *** CITATION PACKAGES ***
%
% \usepackage{cite}
% cite.sty was written by Donald Arseneau
% V1.6 and later of IEEEtran pre-defines the format of the cite.sty package
% \cite{} output to follow that of IEEE. Loading the cite package will
% result in citation numbers being automatically sorted and properly
% "compressed/ranged". e.g., [1], [9], [2], [7], [5], [6] without using
% cite.sty will become [1], [2], [5]--[7], [9] using cite.sty. cite.sty's
% \cite will automatically add leading space, if needed. Use cite.sty's
% noadjust option (cite.sty V3.8 and later) if you want to turn this off
% such as if a citation ever needs to be enclosed in parenthesis.
% cite.sty is already installed on most LaTeX systems. Be sure and use
% version 4.0 (2003-05-27) and later if using hyperref.sty. cite.sty does
% not currently provide for hyperlinked citations.
% The latest version can be obtained at:
% http://www.ctan.org/tex-archive/macros/latex/contrib/cite/
% The documentation is contained in the cite.sty file itself.






% *** GRAPHICS RELATED PACKAGES ***
%
\ifCLASSINFOpdf
  % \usepackage[pdftex]{graphicx}
  % declare the path(s) where your graphic files are
  % \graphicspath{{../pdf/}{../jpeg/}}
  % and their extensions so you won't have to specify these with
  % every instance of \includegraphics
  % \DeclareGraphicsExtensions{.pdf,.jpeg,.png}
\else
  % or other class option (dvipsone, dvipdf, if not using dvips). graphicx
  % will default to the driver specified in the system graphics.cfg if no
  % driver is specified.
  % \usepackage[dvips]{graphicx}
  % declare the path(s) where your graphic files are
  % \graphicspath{{../eps/}}
  % and their extensions so you won't have to specify these with
  % every instance of \includegraphics
  % \DeclareGraphicsExtensions{.eps}
\fi

\begin{document}
%
% paper title
% can use linebreaks \\ within to get better formatting as desired
% Do not put math or special symbols in the title.
\title{Literature Study: Identity Based Encryption for Online Social Networks}
%
%
% author names and IEEE memberships
% note positions of commas and nonbreaking spaces ( ~ ) LaTeX will not break
% a structure at a ~ so this keeps an author's name from being broken across
% two lines.
% use \thanks{} to gain access to the first footnote area
% a separate \thanks must be used for each paragraph as LaTeX2e's \thanks
% was not built to handle multiple paragraphs
%

\author{Stijn Meul,
        ir. Filipe Beato,~\IEEEmembership{Supervisor}
        Prof. dr. ir. Bart Preneel,~\IEEEmembership{Promotor}
        Prof. dr. ir. Vincent Rijmen,~\IEEEmembership{Promotor}%
        }

% The paper headers
\markboth{}%
{}
% The only time the second header will appear is for the odd numbered pages
% after the title page when using the twoside option.
% 
% *** Note that you probably will NOT want to include the author's ***
% *** name in the headers of peer review papers.                   ***
% You can use \ifCLASSOPTIONpeerreview for conditional compilation here if
% you desire.


% make the title area
\maketitle

% As a general rule, do not put math, special symbols or citations
% in the abstract or keywords.
\begin{abstract}
The abstract goes here.
\end{abstract}

% Note that keywords are not normally used for peerreview papers
% \begin{IEEEkeywords}
%IEEEtran, journal, \LaTeX, paper, template.
%\end{IEEEkeywords}






% For peer review papers, you can put extra information on the cover
% page as needed:
% \ifCLASSOPTIONpeerreview
% \begin{center} \bfseries EDICS Category: 3-BBND \end{center}
% \fi
%
% For peerreview papers, this IEEEtran command inserts a page break and
% creates the second title. It will be ignored for other modes.
\IEEEpeerreviewmaketitle



\section{Introduction}
\IEEEPARstart{O}{nline} Social Networks (\textit{OSNs}) are increasingly being
used to share sensitive data with a limited circle of online connections. Users
therefore have to count on the privacy infrastructure of the social network
itself. On social networks like Google Plus or Facebook this is realised by
defining groups of connections who are allowed to see certain profile updates.
At first sight the user seems protected from the general public as it does not
have access to these shielded updates. The OSN itself however still has access
to this data and gratefully uses it for comercial purposes like targeted
advertising or behavior analysis. All this data forms the basis of the OSNs
economical existence. As recent events have shown, this user data is not
only used for the sake of the OSNs financial survival but the data is
 leaked to third parties and agencies like NSA as well. One can wonder if a
company non-transparently passes this data to any government agency for the
argument of security, what other purposes this data will be used for.\\
\\
Because the OSN has access to all users' data, it can be shared with external
parties. This takes the users' full control over their data completely out of
their hands. Additionally OSNs might offer API's to expose the users'
information to other services like external applications. Finally, OSNs
intentionally change security policies to maintain the balance between
usability and the commercial value of their databases thereby leaving their
users privacy behind.~\cite{BeatoScramble}\\
\\
One could argue that if one wants to stay anonymous, he should not use an OSN.
As the increasing popularity of OSNs has shown however, there is a market
demand for OSNs because of their wide range of useful applications. It would
therefore still be meaningful to make use of the infrastructure of a Social
Network without its inherent data leakage.\\
\\
The goal of this literature study is to research an elegant architecture that
allows to keep using OSN infrastructure without having to rely on the
willingness of OSNs to keep the user's data private. The study is done with
particularly Facebook in mind as this social network is amongst the most
popular ones in the world. In the end the user should be able to define his
own privacy instead of being dictated by the social network.



\section{Existing Architectures~\cite{BeatoScramble}}
Lots of fundamental work has been done on applications trying to enforce
access control rules on OSNs. A grasp of all available applications is
listed and analysed in the following section.

\paragraph{flyByNight} is a Facebook application that protects user data by
storing it in encrypted form in Facebook. It relies on Facebook servers for its
key management and is therefore not secure against active attacks by Facebook
itself.

\paragraph{NOYB (None Of Your Business)} replaces the details of user A with
those of random users B and C thereby making this process only reversable by
friends who are allowed to see the profile of user A. However this can not be
applied to messages or status updates that are frequently used on Facebook.

\paragraph{FaceCloak} stores published Facebook data on its servers in encrypted
form and replaces the data on Facebook with random text fetched from Wikipedia.
Although this could be a useful mechanism to prevent OSNs from blocking security
aware users because they are scared to see their advertising revenues shrink, it
has the disadvantage that other users could take this data to be genuine user
content. Furthermore FaceCloaks architecture leads to an inefficient key
distribution system.

\paragraph{Persona} is a scheme that can be used as a Firefox extension to let
users of an OSN determine their own privacy by supporting the ability to encrypt
messages to a group of earlier defined friends based on \textit{attribute-based
encryption (ABE)}\cite{SahaiFuzzyIBE}. The scheme supports lots of useful use
cases such as sending messages to all friends that are related to a certain
attribute or even encrypting messages to friends of friends. The major drawback
of this system however is that every new friend has to exchange a public key
before he is able to interact in the privacy preserving architecture
consequently requiring an infrastructure for broadcasting and storing public
keys. Furthermore, to support the encryption of messages to friends of friends,
user defined groups should be made available publicly thereby making the public
key distribution system even more complicated. Finally the proposed ABE
encryption scheme is 100 to 1000 times slower than a standard RSA operation.
\cite{BadenPersona} 

\paragraph{Scramble} is a Firefox extension that allows defining groups of
friends that are given access to certain social network updates. The tool uses
public key encryption based on OpenPGP \cite{rfc4880} to broadcast encrypted
messages on almost any platform. Furthermore Scramble provides the
implementation of a tiny link server such that OSN policies not allowing to post
encrypted data are bypassed. However, as indicated by usability studies
\cite{WhittenJohnny} OpenPGP has a higher usage threshold because an average
user does not mannage to understand OpenPGP properly. Additionally Scramble has
to rely on the security decisions of the web of thrust. It therefore inherits
the upleasant property of OpenPGP that the user can not be sure that the used
PGP key actually belongs to the intended Facebook profile.\cite{BeatoScramble}\\
\\
The most unattractive property of all the above applications is that they have
to rely on a rather complex infrastructure. Persona has to support an extended
public key distribution system and Scramble relies on the leap-of-faith PGP web
of thrust. Would it not be pleasant to use an infrastructure that inherently
ensures that a Facebook update can only be read by the profiles the user
inteded to? Would it not be a desirable approach that Facebook would require
users to publish a public key on their Facebook profile such that communication
with external providers is limited to a minimum? As the most important part of
an OSNs revenues lies in the data analysis of their users, the chances of
Facebook implementing a required public key option are rather low to
unexisting. To circumvent the requirement of such a public key attribute, it
would be user friendly if any unique public string could be used as a public
key. This is where identity based cryptography comes in.\\
\\
\section{Identity Based Encryption}
Shamir proposed a concept of identity-based cryptography in
1984~\cite{DBLP:conf/crypto/Shamir84}. The basic idea behind identity-based
cryptography is that any string can be a valid public key. Identity-based
cryptography is particularly elegant if the public key is related to an
attribute that uniquely identifies the identity of the user like an e-mail
address or a telephone number~\cite{Baek04asurvey}.

\subsection{Subsection Heading Here}
Subsection text here.

% needed in second column of first page if using \IEEEpubid
%\IEEEpubidadjcol

\subsubsection{Subsubsection Heading Here}
Subsubsection text here.


% An example of a floating figure using the graphicx package.
% Note that \label must occur AFTER (or within) \caption
% For figures, \caption
% should occur after the \includegraphics.
% Note that IEEEtran v1.7 and later has special internal code that
% is designed to preserve the operation of \label within \caption
% even when the captionsoff option is in effect. However, because
% of issues like this, it may be the safest practice to put all your
% \label just after \caption rather than within \caption{}.
%
% Reminder: the "draftcls" or "draftclsnofoot", not "draft", class
% option should be used if it is desired that the figures are to be
% displayed while in draft mode.
%
%\begin{figure}[!t]
%\centering
%\includegraphics[width=2.5in]{myfigure}
% where an .eps filename suffix will be assumed under latex, 
% and a .pdf suffix will be assumed for pdflatex; or what has been declared
% via \DeclareGraphicsExtensions.
%\caption{Simulation Results.}
%\label{fig_sim}
%\end{figure}

% Note that IEEE typically puts floats only at the top, even when this
% results in a large percentage of a column being occupied by floats.

% An example of a double column floating figure using two subfigures.
% (The subfig.sty package must be loaded for this to work.)
% The subfigure \label commands are set within each subfloat command,
% and the \label for the overall figure must come after \caption.
% \hfil is used as a separator to get equal spacing.
% Watch out that the combined width of all the subfigures on a 
% line do not exceed the text width or a line break will occur.
%
%\begin{figure*}[!t]
%\centering
%\subfloat[Case I]{\includegraphics[width=2.5in]{box}%
%\label{fig_first_case}}
%\hfil
%\subfloat[Case II]{\includegraphics[width=2.5in]{box}%
%\label{fig_second_case}}
%\caption{Simulation results.}
%\label{fig_sim}
%\end{figure*}
%
% Note that often IEEE papers with subfigures do not employ subfigure
% captions (using the optional argument to \subfloat[]), but instead will
% reference/describe all of them (a), (b), etc., within the main caption.


% An example of a floating table. Note that, for IEEE style tables, the 
% \caption command should come BEFORE the table. Table text will default to
% \footnotesize as IEEE normally uses this smaller font for tables.
% The \label must come after \caption as always.
%
%\begin{table}[!t]
%% increase table row spacing, adjust to taste
%\renewcommand{\arraystretch}{1.3}
% if using array.sty, it might be a good idea to tweak the value of
% \extrarowheight as needed to properly center the text within the cells
%\caption{An Example of a Table}
%\label{table_example}
%\centering
%% Some packages, such as MDW tools, offer better commands for making tables
%% than the plain LaTeX2e tabular which is used here.
%\begin{tabular}{|c||c|}
%\hline
%One & Two\\
%\hline
%Three & Four\\
%\hline
%\end{tabular}
%\end{table}


% Note that IEEE does not put floats in the very first column - or typically
% anywhere on the first page for that matter. Also, in-text middle ("here")
% positioning is not used. Most IEEE journals use top floats exclusively.
% Note that, LaTeX2e, unlike IEEE journals, places footnotes above bottom
% floats. This can be corrected via the \fnbelowfloat command of the
% stfloats package.



\section{Conclusion}
The conclusion goes here.
% if have a single appendix:
%\appendix[Proof of the Zonklar Equations]
% or
%\appendix  % for no appendix heading
% do not use \section anymore after \appendix, only \section*
% is possibly needed

% use appendices with more than one appendix
% then use \section to start each appendix
% you must declare a \section before using any
% \subsection or using \label (\appendices by itself
% starts a section numbered zero.)
%



\appendices
\section{Proof of the First Zonklar Equation}
Appendix one text goes here.

% you can choose not to have a title for an appendix
% if you want by leaving the argument blank
\section{}
Appendix two text goes here


% use section* for acknowledgement
\section*{Acknowledgment}


The authors would like to thank..


% Can use something like this to put references on a page
% by themselves when using endfloat and the captionsoff option.
\ifCLASSOPTIONcaptionsoff
  \newpage
\fi


% trigger a \newpage just before the given reference
% number - used to balance the columns on the last page
% adjust value as needed - may need to be readjusted if
% the document is modified later
%\IEEEtriggeratref{8}
% The "triggered" command can be changed if desired:
%\IEEEtriggercmd{\enlargethispage{-5in}}

% references section

% can use a bibliography generated by BibTeX as a .bbl file
% BibTeX documentation can be easily obtained at:
% http://www.ctan.org/tex-archive/biblio/bibtex/contrib/doc/
% The IEEEtran BibTeX style support page is at:
% http://www.michaelshell.org/tex/ieeetran/bibtex/
% \bibliographystyle{IEEEtran}
% argument is your BibTeX string definitions and bibliography database(s)
%\bibliography{IEEEabrv,../bib/paper}
%
% <OR> manually copy in the resultant .bbl filer
% set second argument of \begin to the number of references
% (used to reserve space for the reference number labels box)
\bibliography{references}




%\begin{thebibliography}{1}
%\bibitem{IEEEhowto:kopka}
%H.~Kopka and P.~W. Daly, \emph{A Guide to \LaTeX}, 3rd~ed.\hskip 1em plus
%  0.5em minus 0.4em\relax Harlow, England: Addison-Wesley, 1999.
%\bibitem{}

%\end{thebibliography}



% that's all folks
\end{document}
